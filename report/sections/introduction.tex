With cities supporting more extensive diversity than ever in modes of last-mile transportation,
having quantifiable data on cyclist movement is a crucial element in modern city planning. 
Movement patterns of cyclists and other traffic participants and the intersection in-between 
is to a large degree determined by the design of the shared infrastructure. 
Traditionally, traffic behavior analysis has been dependent on manual human review, either by observing locations on-site or through recorded footage. 
This approach comes at a high cost, and unless rigid methodology is defined beforehand, reports are often ad-hoc and qualitative. 
\ \\

\color{red}
Signaled intersections with mixed .
\color{black}

While bicycle paths at intersections do increase cyclists' perception of safety, 
studies conducted in Denmark show a significant increase in accidents after the installation of bicycle paths (\cite{intersection_safety}).
We thus aim to build a system that can automatically capture the behavioral patterns of cyclists at intersections.
Using recorded video footage from stationary cameras and applying modern object detection algorithms, 
we can extract the trajectories of road users. 
This in turn can help identify the relationships between behavior and infrastructure and their effect on safety. 
\ \\

A considerable benefit of implementing such a system comes from the ability to scale and standardize measurements.
As both cameras and hardware for graphical computation have become readily available, such an approach is now financially viable.