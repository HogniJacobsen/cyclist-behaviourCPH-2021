\color{red}
Firstly we cover the technical advances in object detection over the past decade and its usage in traffic monitoring.
\color{black}

\subsection{Staging mobilities}
The book 'Designing Mobilities' (\cite{designinig_mobilities}) explores \textit{what physical, social, technical, and cultural conditions contribute to the staging of contemporary urban mobilities?} It covers how to capture and represent \textit{mobilities} and flows by presenting a theoretical framework for 'staging mobilities'. 
Most importantly it introduces a set of concepts for articulating \textit{situational perspectives},
 including the two metaphors \textit{the river} and \textit{the ballet}. 
 \ \\

 The \textit{river}references the notion of capturing mobility from a 'birds-eye view' including the layout, flows, and obstacles of a location, and hence, can be interpreted as a 'flowing riverbed'. From this perspective, we can observe road users' behavior in aggregate and as part of more significant 'streams'. This can be used to encapsulate abstract and high-level features of cyclists, such as the change in paths
 when infrastructure is changed or temporary obstacles come along (e.g., road work).
 \ \\

In contrast, the \textit{ballet} is the micro-perspective, encompassing the gestures and small interactions at the individual level.
This perspective is an important supplement to the former \textit{river} as subtle interactional patterns which can, for instance,
 signal agreement or evasion.

\subsubsection{Case studies}
\textit{Study of mobilities in 'situ'} (\cite{situ}) uses the above framework as part of a case study involving cycling infrastructure in 
Copenhagen and Amsterdam. Quantitative data (as part of the \textit{river} perspective) was obtained by filming cyclists and conducting 
"desire lines analysis" (\cite{cva}) to gain a structured overview of the behavior of cyclists and intersections. 

\color{red}
More on this.
\color{black}

\color{red}
\subsection{Inclusive infrastructure}
Why it is crucial to understand cyclist behavior, and especially in cities with emerging bike infrastructure, \textit{how they adapt}.
We focus on the psychological aspects of a cyclist,

\begin{itemize}
	\item Perception of safety
	\item Intuitiveness (right-of-way should be obvious)
	\item Congestion (incl. waiting times)
	\item Rule compliance
\end{itemize}

If new cyclists feel uncomfortable (stress, embarrassment, etc.) they are more likely to give up on the bike. This leaves the infrastructure 
under-used, which can further result in a back-clash against "bike culture". 