Firstly we cover the technical advances in object detection over the past decade and its usage in traffic monitoring.

\subsection{Desire lines}
We look at how bicycle infrastructure (especially intersections) can be assessed. Here we'll mostly use 
trajectories and the idea of "desire lines" are metrics to focus on. (\cite{situ}). 



\subsection{Inclusive infrastructure}
Why it's important to understand cyclist behavior, and especially in cities with emerging bike infrastructure, \textit{how they adapt}.
We focus on the psychological aspects of a cyclist,

\begin{itemize}
	\item Perception of safety
	\item Intuitiveness (right-of-way should be obvious)
	\item Congestion (incl. waiting times)
	\item Rule compliance
\end{itemize}

If new cyclists feel uncomfortable (stress, embarrassment etc.) they're more likely to give up on the bike. This leaves the infrastructure 
under-used, which can further result in a back-clash against "bike culture".