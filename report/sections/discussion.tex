\subsection{Privacy}

Consideration should be taken towards privacy and the legality of collecting and processing data.
\ \\

In the European Union (EU), where major cycling hubs of the world lie, we are governed by the General Data Protection Regulation (GDPR). GDPR regulates the processing of personal data 
relating to individuals by individuals, companies, and organizations.
\ \\

Article 6 of the EU GDPR deals with the 'Lawfulness of processing'. Specifically Article 6(1)(f)
makes provision for legitimate interests, which could cover the scope of this research.
However, this is not the only consideration. Local authorities shall be consulted before commencing any research.
\ \\

\subsection{Limitations}
\subsubsection{Hardware}
The Galaxy S7 is in this methodology does not live up to our criteria in terms of its 
ability to record the entire intersection from its mounting location. While it captured an adequate 
FOV we could produce better results with a camera that is up to standard. The choice to use this device 
was due to only receiving a different device too late in the process.
\ \\

\subsubsection{Multiple object tracking with joined data}
We currently have no method, other than visual inspection, of quantifying how well the multiple object 
detection works along the joining line of the two videos.
\ \\

\subsubsection{Small and sparse sample size}
We only used 1.5 hours of video from the intersection, which is a relatively small sample size considering
that previous analyses have used up to 12 hours of video. The videos were also recorded at 12 PM on a Wednesday 
which is a period of low congestion. While we have seen promising object identification results with small groups
of cyclists it is unclear as to how object detection would perform on higher congested intersections. 
\ \\

\subsection{Future work}
Future work could implement a method of clustering trajectories,
such as in \cite{gariel_trajectory_2011}. Clustering could enable a more fine-grained examination of desire lines,
along with greater metadata such as counts per desire path.
\ \\

The SORT method of multiple object detection could be replaced with its extended version \href{https://github.com/nwojke/deep_sort}{Deep SORT}.
Deep SORT extends on SORT by, in addition to IOU as a linking criteria, also incorporating
deep features of the objects for linking. Deep SORT could greatly improve the multiple object tracking
and paving way for an implementation that uses more than two cameras.
\ \\

Rule compliance and unintended behavior are tightly coupled with the state of the traffic signal at intersections. 
Having functionality to mark traffic lights and have the ability to filter results based on their state would
be a useful addition. \textit{more on RUBA at AAU}